\documentclass{article}  % Define la clase del documento.

% Paquetes de idioma y codificación
\usepackage[utf8]{inputenc}
\usepackage[T1]{fontenc}
\usepackage[spanish]{babel}  % Ajusta el idioma del documento a español.

% Paquete de geometría para configurar márgenes y tamaño de papel
\usepackage[letterpaper, margin=3cm]{geometry}

% Paquetes de tipografía
\usepackage{mathptmx}    % Usa Times New Roman como fuente.
\usepackage{microtype}   % Mejora la justificación del texto.

% Paquetes para manejo de colores y gráficos
\usepackage{xcolor}      % Define y utiliza colores.
\usepackage{graphicx}    % Permite la inserción de imágenes.
\usepackage{tikz}        % Creación de gráficos vectoriales.

% Configuración de enlaces y referencias cruzadas
\usepackage{hyperref}
\hypersetup{
    colorlinks   = true,
    linkcolor    = darkblue,
    citecolor    = black,
    filecolor    = blue,
    urlcolor     = blue
}

% Paquetes para la mejora visual de tablas y figuras
\usepackage{booktabs}    % Para tablas de alta calidad.
\usepackage{float}       % Controla la posición de figuras y tablas.

% Paquete para la personalización de códigos fuente
\usepackage{listings}
\lstset{
    literate=
    {á}{{\'a}}1 {é}{{\'e}}1 {í}{{\'i}}1 {ó}{{\'o}}1 {ú}{{\'u}}1
    {Á}{{\'A}}1 {É}{{\'E}}1 {Í}{{\'I}}1 {Ó}{{\'O}}1 {Ú}{{\'U}}1
    {ñ}{{\~n}}1 {Ñ}{{\~N}}1 {ü}{{\"u}}1 {Ü}{{\"U}}1,
    backgroundcolor=\color{backcolour},
    commentstyle=\color{codegreen},
    keywordstyle=\color{codepurple},
    numberstyle=\tiny\color{codegray},
    stringstyle=\color{red},
    basicstyle=\ttfamily\small,
    breakatwhitespace=false,
    breaklines=true,
    captionpos=b,
    keepspaces=true,
    numbers=left,
    numbersep=5pt,
    showspaces=false,
    showstringspaces=false,
    showtabs=false,
    tabsize=2,
    language=TeX,
    morecomment=[l]\#,
    frame=single,
    rulecolor=\color{black}
}

% Definición de colores al estilo Visual Studio Code
\definecolor{darkblue}{rgb}{0.0, 0.0, 0.55}  % Enlaces
\definecolor{codegreen}{rgb}{0.25, 0.49, 0.48}  % Comentarios
\definecolor{codegray}{rgb}{0.5, 0.5, 0.5}  % Números y anotaciones
\definecolor{codepurple}{rgb}{0.58, 0, 0.82}  % Palabras clave
\definecolor{backcolour}{rgb}{0.95, 0.95, 0.92}  % Fondo de código

% Configuraciones de párrafo y matemáticas
\usepackage{amsmath}
\usepackage{parskip}    % Espaciado entre párrafos.
\usepackage{ragged2e}   % Justificación mejorada.

% Configuración de secciones y encabezados
\usepackage{titlesec}
\titleclass{\part}{top} % Make part like a class
\titleformat{\part}[display]
  {\normalfont\huge\bfseries\centering}{\thepart}{20pt}{\Huge}
\titlespacing*{\part}{172.5pt}{-60pt}{10pt}
\titleformat{\part}
  {\normalfont\huge\bfseries}{}{0pt}{}

% Asegúrate de usar esto para mantener el estilo en las páginas de las partes
\titleformat{\part}[display]
  {\normalfont\huge\bfseries}{}{0pt}{}
  [\thispagestyle{fancy}] % Aplica el estilo fancy a las páginas de las partes

% Configuración de encabezados y pies de página personalizados
\usepackage{fancyhdr}
\pagestyle{fancy}
\fancyhf{}
\fancyhead[L]{\raisebox{0.20cm}{\textbf{Hidrología}}}
\fancyhead[R]{\raisebox{0.1cm}{\includegraphics[width=0.25\linewidth]{LOGO_UNIVERSIDAD.jpg}}}
\fancyhead[C]{\rule{\textwidth}{0.6pt}}
\fancyfoot[C]{\rule{\textwidth}{0.6pt}}
\fancyfoot[R]{\raisebox{-1.5\baselineskip}{\thepage}}
\renewcommand{\headrulewidth}{0pt}
\renewcommand{\footrulewidth}{0pt}

% Configuración avanzada de geometría
\geometry{
  top=3.5cm, % Aumenta el espacio en la parte superior para subir el encabezado
  bottom=2.5cm,
  headheight=2.5cm % Aumenta la altura del encabezado si es necesario
}

% Configuracion de bibliografia
\usepackage{natbib}
\bibliographystyle{unsrtnat}  % Puedes cambiarlo por `unsrtnat`, `abbrvnat`, etc.

\begin{document}
%----------------------------------------------------------------------------------------
% PORTADA
%----------------------------------------------------------------------------------------
\begin{titlepage}%Inicio de la carátula, solo modificar los datos necesarios
\newcommand{\HRule}{\rule{\linewidth}{0.5mm}} 
\center 
%----------------------------------------------------------------------------------------
%	ENCABEZADO
%----------------------------------------------------------------------------------------
\includegraphics[width=10cm]{LOGO_UNIVERSIDAD.jpg}\\ % Si esta plantilla se copio correctamente, va a llevar la imagen del logo de la facultad.OBS: Es necesario incluir el paquete: graphicx
\vspace{3cm}
%----------------------------------------------------------------------------------------
%	SECCION DEL TITULO
%----------------------------------------------------------------------------------------
\HRule \\[0.4cm]
{ \huge \bfseries Tarea 2}\\[0.4cm] % Titulo del documento
{ \huge \bfseries Hidrología}\\[0.4cm] % Titulo del documento
\HRule \\[1.5cm]
 \vspace{5cm}
%----------------------------------------------------------------------------------------
%	SECCION DEL AUTOR
%----------------------------------------------------------------------------------------
\begin{flushright}
    { \textbf{Profesor:}\\
    Ricardo González \\
    \vspace{0.2cm}
    \textbf{Alumno:} \\
    Bernardo Caprile \\
    Pedro Valenzuela \\
    Felipe Vicencio \\
    Lukas Wolff \\
}
\end{flushright}
\vspace{1cm}
%----------------------------------------------------------------------------------------
%	SECCION DE LA FECHA
%----------------------------------------------------------------------------------------
{\large \textbf{\today}}\\[2cm] % El comando \today coloca la fecha del dia, y esto se actualiza con cada compilacion, en caso de querer tener una fecha estatica, reemplazar el \today por la fecha deseada
\end{titlepage}
%----------------------------------------------------------------------------------------
%  INDICE
%----------------------------------------------------------------------------------------
\newpage
\thispagestyle{empty} % Deshabilita el número de página en la página del índice
\tableofcontents
\thispagestyle{plain} % Deshabilita el encabezado en la página del índice
\thispagestyle{empty} % Deshabilita el número de página en la página del índice
\newpage

%----------------------------------------------------------------------------------------
%ACÁ EMPIEZA EL INFORME
\setcounter{page}{1}
%----------------------------------------------------------------------------------------
\section{Introducción}
HACER INTRO
\newpage
\section{Resultados}

\subsection{Pregunta 1}

\subsubsection{Marco Teórico}

La presión de saturación del vapor de agua se determina mediante la ecuación de Clausius-Clapeyron:
  
\begin{equation}
  e_s(T) = 611 \cdot e^{\left(\frac{17.27 \cdot T}{T + 237.3}\right)}
\end{equation}

La humedad relativa se determina mediante la siguiente ecuación:

\begin{equation}
  RH = \frac{e}{e_s} \cdot 100
\end{equation}

La radiación neta y emitida se determinan mediante las siguientes ecuaciones:

\begin{equation}
  Re = \varepsilon \sigma T^4
\end{equation}

\begin{equation}
  R_n = R_i \cdot (1 - \alpha) - R_{e}
\end{equation}

Donde:

\begin{itemize}
  \item $R_n$ es la radiación neta.
  \item $R_i$ es la radiación incidente.
  \item $\alpha$ es el albedo.
  \item $R_e$ es la radiación emitida.
\end{itemize}

Para obtener la evaporación con el método aerodinámico se usan las siguientes ecuaciones:

\begin{equation}
  E_a = B(e_{as} - e_a)
\end{equation}
  
\begin{equation}
  B = \frac{0.622 k^2 \rho_a u_2}{P \rho_w [\ln(Z_2 / Z_o)]^2}
\end{equation}

Donde:

\begin{itemize}
  \item $E_a$ es la evaporación.
  \item $B$ es el coeficiente de evaporación.
  \item $e_{as}$ es la presión de saturación del vapor de agua.
  \item $e_a$ es la presión de vapor de agua.
  \item $k$ es la constante de von Karman.
  \item $\rho_a$ es la densidad del aire.
  \item $u_2$ es la velocidad del viento a 2 metros de altura.
  \item $P$ es la presión atmosférica.
  \item $\rho_w$ es la densidad del agua.
  \item $Z_2$ es la altura a la que se mide la velocidad del viento.
  \item $Z_o$ es la altura de rugosidad.
\end{itemize}

Por otro lado, para obtener la evaporación con el método de balance de energía se usó lo siguiente:

\begin{equation}
  E_r = \frac{R_n}{l_v \rho_w}
\end{equation}

Donde:

\begin{itemize}
  \item $E_r$ es la evaporación.
  \item $l_v$ es el calor latente de vaporización.
  \item $\rho_w$ es la densidad del agua.
\end{itemize}

\subsubsection{Resultados}

A continuación, se presentan los datos iniciales del problema.

\begin{table}[H]
  \centering
  \caption{Datos iniciales del problema.}
  \begin{tabular}{|c|c|c|}
    \hline
    \textbf{Variable} & \textbf{Valor} & \textbf{Unidad}\\
    \hline
    $T_{agua}$ & 10 & °C\\
    $T_{aire}$ & 20 & °C\\
    $RH$ & 65 & \%\\
    $R_i$ & 455,67 & $\frac{W}{m^2}$\\
    $\alpha$ & 0.05 & - \\
    $V_{viento}$ & 2,5 & $\frac{m}{s}$\\
    $P_{atm}$ & 101 & kPa\\
    $Z_2$ & 2.0 & m\\
    $Z_o$ & 0.0003 & m\\
    $l_v$ & $2.45 \times 10^6$ & $\frac{J}{kg}$\\
    $\epsilon$ & 0,97 & -\\
    $\sigma$ & $5.67 \times 10^{-8}$ & $\frac{W}{m^2K^4}$\\
    $\rho_{aire}$ & 1,201 & $\frac{kg}{m^3}$\\
    \hline
  \end{tabular}
  \\ Fuente: elaboración propia.
\end{table}

Aplicando las ecuaciones y procedimientos expuestos en el marco teórico, se llegaron a los siguientes resultados.

\begin{table}[H]
  \centering
  \caption{Resultados de ambos métodos de cálculo.}
  \begin{tabular}{|c|c|c|}
      \hline
      Variable & Aerodinámico & Balance Energético \\ \hline
      $R_e$ & \multicolumn{2}{|c|}{$353,52 \frac{W}{m^2}$} \\ \hline
      $R_{neta}$ & \multicolumn{2}{|c|}{$79,35 \frac{W}{m^2}$} \\ \hline
      $e_s$ & $2,34 kPa$ & - \\ \hline
      $e_a$ & $1,52 kPa$ & - \\ \hline
      $B$ & $4,017 \times 10^{-8}$ & - \\ \hline
      $E_v$ & $2,84 mm/dia$ & $2,80 mm/dia$ \\ \hline
      $Error$ & \multicolumn{2}{|c|}{$1,51 \%$} \\ \hline
  \end{tabular}
  \\ Fuente: elaboración propia.
\end{table}

\subsection{Pregunta 2}

\subsubsection{Marco Teórico}

La evapotranspiración de referencia se calcula utilizando la ecuación de Penman-Monteith, que estima la cantidad de agua que un cultivo podría perder por evaporación y transpiración bajo condiciones estándar:

\begin{equation}
  ETo = \frac{0.408 \Delta (R_n - G) + \gamma \frac{900}{T_{mean} + 273} U_2 (e_s - e_a)}{\Delta + \gamma (1 + 0.34 U_2)}
\end{equation}

Donde:

\begin{itemize}
  \item $ETo$ es la evapotranspiración de referencia (mm/día).
  \item $\Delta$ es la pendiente de la curva de presión de vapor (kPa/°C).
  \item $R_n$ es la radiación neta en la superficie del cultivo (MJ/m²/día).
  \item $G$ es el flujo de calor del suelo (MJ/m²/día).
  \item $\gamma$ es la constante psicrométrica (kPa/°C).
  \item $T_{mean}$ es la temperatura media diaria (°C).
  \item $U_2$ es la velocidad del viento a 2 metros de altura (m/s).
  \item $e_s$ es la presión de vapor de saturación (kPa).
  \item $e_a$ es la presión de vapor real (kPa).
\end{itemize}

La presión de vapor de saturación ($e_s$) se determina mediante la siguiente ecuación:

\begin{equation}
  e_s = 0.6108 \exp \left( \frac{17.27 T_{mean}}{T_{mean} + 237.3} \right)
\end{equation}

Donde:

\begin{itemize}
  \item $T_{mean}$ es la temperatura media diaria (°C).
\end{itemize}

La presión de vapor real ($e_a$) se calcula multiplicando la presión de vapor de saturación por la humedad relativa:

\begin{equation}
  e_a = e_s \left( \frac{HR}{100} \right)
\end{equation}

Para calcular la tasa de riego, se usa la siguiente ecuación:

\begin{equation}
  TR = \frac{ET_c - P_u - H}{n}
\end{equation}

Donde:

\begin{itemize}
  \item $TR$ es la tasa de riego (mm/día).
  \item $ET_c$ es la evapotranspiración del cultivo en el mes de máxima demanda (mm/día).
  \item $P_u$ es la precipitación efectiva (mm/día).
  \item $H$ es la humedad del suelo (mm/día).
  \item $n$ es la eficiencia del sistema de riego.
\end{itemize}

Finalmente, la tasa de riego se convierte en caudal utilizando la siguiente expresión:

\begin{equation}
  Q = \frac{TR \cdot A}{3600}
\end{equation}

Donde:

\begin{itemize}
  \item $Q$ es el caudal de diseño del sistema de riego (m³/s).
  \item $TR$ es la tasa de riego (m/día).
  \item $A$ es el área del cultivo (m²).
  \item $3600$ es el número de segundos en una hora.
\end{itemize}

\subsubsection{Resultados}

A continuación, se presentan los datos iniciales del problema.

\begin{table}[H]
  \centering
  \begin{tabular}{|l|l|c|c|c|c|c|c|c|c|c|c|c|c|}
  \hline
  \textbf{Param} & \textbf{Unidad} & \textbf{Ene} & \textbf{Feb} & \textbf{Mar} & \textbf{Abr} & \textbf{May} & \textbf{Jun} & \textbf{Jul} & \textbf{Ago} & \textbf{Sep} & \textbf{Oct} & \textbf{Nov} & \textbf{Dic} \\ \hline
  U2    & km/día       & 243 & 240 & 120 & 118 & 115 & 242 & 242 & 121 & 242 & 242 & 242 & 242 \\ \hline
  Rn    & MJ/m²/día    & 25.8 & 18.6 & 11.4 & 11.9 & 7.9 & 6.6 & 7.6 & 10.4 & 11.9 & 17 & 20.2 & 24.7 \\ \hline
  Tmin  & °C           & 11.6 & 11.3 & 9.8 & 8.1 & 7.4 & 6.0 & 5.3 & 5.8 & 7.0 & 8.2 & 9.2 & 10.7 \\ \hline
  Tmax  & °C           & 26.7 & 26.4 & 25.6 & 22.6 & 19.2 & 16.7 & 16.7 & 18.1 & 19.6 & 21.8 & 25.4 & 26.2 \\ \hline
  HR    & \%           & 71 & 74 & 74 & 76 & 79 & 79 & 79 & 79 & 79 & 75 & 72 & 70 \\ \hline
  \end{tabular}
  \caption{Parámetros meteorológicos de la estación de Quillota}
\end{table}

Aplicando las ecuaciones y procedimientos expuestos en el marco teórico, se llegaron a los siguientes resultados.

Los resultados obtenidos para la evapotranspiración de referencia (ETo) en cada mes del año son los siguientes (en mm/día):

\begin{table}[H]
\centering
\begin{tabular}{|l|c|}
\hline
\textbf{Mes} & \textbf{ETo (mm/día)} \\ \hline
Enero        & 3.57                  \\ \hline
Febrero      & 2.96                  \\ \hline
Marzo        & 1.86                  \\ \hline
Abril        & 1.56                  \\ \hline
Mayo         & 1.11                  \\ \hline
Junio        & 1.35                  \\ \hline
Julio        & 1.35                  \\ \hline
Agosto       & 1.14                  \\ \hline
Septiembre   & 1.67                  \\ \hline
Octubre      & 2.28                  \\ \hline
Noviembre    & 2.95                  \\ \hline
Diciembre    & 3.48                  \\ \hline
\end{tabular}
\caption{Evapotranspiración de referencia (ETo) calculada para cada mes del año.}
\end{table}

El valor máximo de la evapotranspiración del cultivo (ETc), considerando el coeficiente de cultivo medio ($Kc_{med} = 0.7$), es:

\begin{equation}
ETc = 2.63 \, \text{mm/día}
\end{equation}

Finalmente, la tasa de riego calculada y expresada como caudal es:

\begin{equation}
TR = 5.85 \, \text{m}^3/\text{s}
\end{equation}

  





\end{document}
